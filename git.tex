Dit gedeelte is bedoeld om je bekend te maken met het distributed
version control system {\tt git}. We maken daarbij gebruik van
bestaande online materialen. \\

\stepreset
  
\step Lees \url{http://progit.org/book/ch1-3.html}. Deze pagina vertelt
de beginselen van {\tt git}. \\

\step We gaan nu gebruik maken van Github. Ga naar \url{https://github.com/plans} en maak een gratis - dat
wil zeggen een open source - account aan. Als je dit practicum met zijn
tweeën doet, maak dan allebei een eigen account aan, bij voorkeur op
gescheiden systemen. Als je alleen werkt, maak dan twee accounts aan.
Je hebt dan wel twee verschillende e-mailadressen nodig\footnote{Met
  Gmail kun je dit faken door {\tt +\ldots} achter je e-mailnaam te zetten.
  Bijvoorbeeld: {\tt michielborkent+student1@gmail.com}}. Dit is belangrijk verderop in
het practicum. Je moet dan namelijk iemand toe kunnen voegen als
collaborator aan jouw repository. Het is handig als je jouw
practicumpartner dan kan toevoegen, dus moet hij of zij een
ook een Github-account hebben.\\

\step Installeer daarna {\tt git}. Volg daarvoor het volgende stappenplan:
\url{http://help.github.com/win-set-up-git/}. Via deze link zijn ook
manuals voor linux of Mac OS X beschikbaar. Het is belangrijk dat je
ook het gedeelte over ssh-keys voltooit! Er wordt van een ssh-key
gebruik gemaakt bij het uploaden, of beter in {\tt git}-terminologie:
pushen naar Github. Gebruik bij het aanmaken van
de ssh-key hetzelfde e-mailadres als waarmee je je hebt aangemeld bij
Github. \\

\step In deze stap wordt uitgegaan van Windows. Als je al een linux-
of Mac OS X gebruiker bent, dan kan je deze stap waarschijnlijk
overslaan. Open ``Git Bash'' (hierna kortweg bash genoemd) via Start Menu. Om met deze console te kunnen
werken moet je wel enkele Unix-zaken c.q. Bash-commando's kennen.
Probeer er eens wat uit.

\begin{itemize}
\item Directories worden genoteerd met een / ipv een \textbackslash 
\item  Het station (in Windows bijvoorbeeld {\tt c:} of {\tt d:})
  wordt aangegeven met {\tt /<station>}. Dus de directory {\tt c:\textbackslash
    projects\textbackslash les3} wordt in bash
  weergegeven als {\tt /c/projects/les3}.
\item Het commando {\tt mkdir}: een directory aanmaken. Bijvoorbeeld
  {\tt mkdir /c/projects}.
\item Het commando {\tt cd}, dat is ``change directory''. Bijvoorbeeld  {\tt cd /c/projects/les3}
\item {\tt ls}: opvragen inhoud directory ({\tt dir} in cmd.exe)
\item {\tt rm}: een bestand verwijderen ({\tt del} in cmd.exe)
\end{itemize}

Zie verder deze lijst met bash-commando's: \url{http://ss64.com/bash/}.
Daarnaast is het handig om te weten dat als je op het icoontje in de
linkerbovenhoek van ``Git Bash'' klikt, je kunt knippen, plakken, etc.
Gebruik de tab-toets voor auto-completion.\\

%\step Lees hier hoe je je username en e-mail in kunt stellen: \url{http://progit.org/book/ch1-5.htm%l}. Deze
%instellingen zijn van belang bij ``commit messages''. De rest
%mag je voor nu negeren, tenzij je zelf behoefte hebt om nog meer te
%configureren. 

%\step Maak een directory aan waar je zometeen je git-projecten neer
%kan zetten, bv: {\tt mkdir /c/projects} en dan {\tt
%    mkdir /c/projects/project1}. \\

\step Kies op de Github-website (als je bent ingelogd) ``New Repository''. Vul bij ``Project Name'' in: ``les3''
en bij ``Description'': ``test voor les 3''. Kies vervolgens ``Create
Repository''. Als het goed is krijg je nu een soortgelijk scherm te
zien als in figuur \ref{fig:github-new}. \\

\begin{figure} 
  \caption{Github}
  \includegraphics[width=\textwidth]{pictures/github-new-project.png}
  \label{fig:github-new}
\end{figure}

\step Voer nu de stappen uit die in het scherm staan genoemd, in bash. Ga alvorens dat te doen even naar de directory waar je je
programmeerprojecten bewaard met {\tt cd}. In die directory ga je dan zometeen een
directory genaamd {\tt les3 } maken tijdens het uitvoeren van de
stappen ({\tt mkdir les3}). \\

Uitleg bij de stappen.
\begin{itemize}
  \item {\tt git init} - hiermee maak je van de huidige directory een
    git repository. Tevens wordt er een {\tt .git}-directory
    aangemaakt in {\tt les3} waarin allerlei informatie staat. 
    Er staat nu nog niks in de repository. Als er
    al wel files in de directory stonden, zijn deze niet automatisch
    toegevoegd aan de git repository. Git kent, zoals je in de
    introductie hebt kunnen lezen drie niveaus. Zie ook figuur \ref{fig:github-areas}.
    \item {\tt touch README} - hiermee maak je een (lege) file aan met
      de naam {\tt README}. Dat zou je kunnen controleren door de
      inhoud van de directory op te vragen met {\tt ls}.
    \item {\tt git add README} - hiermee voeg je de file {\tt README}
      toe aan de staging area (zie weer figuur
      \ref{fig:github-areas}). Alles wat is toegevoegd aan de staging
      area zal met een commit worden toegevoegd aan de git repository.
    \item {\tt git commit -m 'first commit'} - hier wordt er een
      daadwerkelijke commit gedaan met als commit message ``first 
      commit''.
      \item {\tt git remote add origin git@github.com:\ldots/les3.git}
        - als remote repository, genaamd {\tt origin} (je had ook een andere
        naam kunnen kiezen), wordt de repository toegevoegd die je net
        op github.com hebt aangemaakt.
      \item {\tt push -u origin master} - verstuur de inhoud van onze
        {\tt master} branch (dit is de branch waarin we nu werken)
        naar de remote repository {\tt origin}. {\tt -u} zorgt ervoor
        dat we de volgende keer simpelweg: {\tt push} kunnen typen om
        de inhoud van de {\tt master} branch naar {\tt origin} te sturen.
\end{itemize}

\begin{figure} 
\begin{center}
  \caption{Git: drie niveaus. (\href{http://progit.org/book/ch1-3.html}{Bron})}
  \includegraphics[width=0.5\textwidth]{pictures/git-areas.png}
  \label{fig:github-areas}
\end{center}
\end{figure}

\step Je kunt bij git-commando's documentatie raadplegen. Als je
bijvoorbeeld meer wil weten over {\tt git push} in bovenstaande lijst
kun je dat opzoeken in de documentatie van {\tt git push}. 
Typ om die documentatie op te vragen
{\tt git help push} in in de console, gevolgd door een enter. Er opent
zich nu een webbrowser met de gevraagde documentatie. Als je de
algemene documentatie van {\tt git} wil raadplegen, typ je simpelweg {\tt
  git help} \\

\step Bekijk nu op Github het resultaat van je eerste push! Dit zou
zichtbaar moeten zijn onder de volgende url:
\url{https://github.com/<username>/les3} met voor {\tt <username>}
jouw username ingevuld. \\

\step Klik op het tabje ``Commits''. Daar vind je een SHA-key van jouw
commit. Er zijn vier soorten objecten in {\tt git}: blobs (files), trees
(directories), commits en tags. Al deze vier soorten objecten worden
geïdentificeerd aan de hand van een unieke SHA1-hash\footnote{Voor meer info
  over SHA1 zie \url{http://en.wikipedia.org/wiki/SHA-1}}. Lees nu het
gedeelte over het object-model van {\tt git}:
\url{http://book.git-scm.com/1_the_git_object_model.html}. \\

\step Kopieer de SHA1-hash (vanaf nu kortweg sha genoemd) van je
eerste commit en typ in bash: {\tt git show <sha>}, waarbij je je
eigen sha invult, bijvoorbeeld: {\tt git show
  d3d01ef8fbaac87f9a3cedfc2839a132b2740937}. {\tt Git} laat nu informatie
zien over je eerste commit. \\

\step Lees nu het gedeelte over de git-index. {\tt Git} maakt onderscheid
tussen de ``working directory'', de ``index'' of ``staging area'' en
de ``repository'' (zie nogmaals \ref{fig:github-areas}!). Wijzigingen
in de working directory worden niet automatisch meegenomen in een
commit. Je moet ze eerst toevoegen aan de index. Met behulp van het
commando {\tt git status} kan je zien wat er in de index staat en dus
klaar staat om gecommit te worden. \\

\step Maak drie nieuwe files aan door {\tt touch file1 file2 file3} te
typen. Maak wat wijzigingen in deze files met bijvoorbeeld Notepad.
Typ vervolgens {\tt git status}. Git komt nu met de melding: {\tt
  nothing added to commit but untracked files present}. Oftewel: bij
een volgende commit zal er niets worden ingecheckt in de repository,
maar er staan wel wijzigingen in de working directory.
Je zal de wijzigingen (nieuwe files zijn ook wijzigingen) eerst moeten
toevoegen aan de index. \\

\step Typ {\tt git add file1 file2 file3}. Probeer nu nogmaals {\tt
  git status}. Git komt nu met een beschrijving van wat er toegevoegd
is aan de index: drie nieuwe files. Ook vertelt git ons hoe we de
index kunnen terugzetten naar {\tt HEAD}. {\tt HEAD} is een afkorting
voor de sha die
correspondeert met onze laatste commit. Dit kan je ook zien aan de
hand van het volgende commando wat de sha laat zien van {\tt HEAD}:
{\tt git rev-parse HEAD}. Dit levert de sha op van onze laatste
commit. \\

\step Typ {\tt git reset HEAD}. We draaien de index terug naar het
moment van onze laatste commit. Bestaan file1, file2 en file3 nog en
is hun inhoud nog intact? Leg uit hoe dit komt. \\

\step We gaan nu weer opnieuw de files toevoegen aan de index, om ze
vervolgens te committen. Dit zal dan onze tweede commit worden. Je kan
alle nieuwe files en wijzigingen in die files toevoegen door {\tt git
  add .} te typen. Typ daarna {\tt git status} om te controleren wat
er zich in de staging area bevindt.

\step Typ nu {\tt git commit -m 'drie nieuwe files toevoegd'}. Met
{\tt -m} geef je een commit message mee. Als je dit niet doet, opent
er een editor waarin je de commit message moet intypen. Standaard
is dat de editor VIM. Als je een andere editor wilt gebruiken moet je
dit instellen. Bijvoorbeeld voor Notepad: {\tt git config
  --global core.editor notepad}. Typ
nogmaals {\tt git rev-parse HEAD}. Zie je dat de sha van HEAD nu
gewijzigd is? Deze sha is de identifier van onze laatste commit. 
Je zou nu een push naar Github kunnen doen. Weet je nog hoe dit moest? \\

\step Grafische weergave. Typ nu in bash {\tt gitk}. Je ziet
nu een grafische weergave van de history. Ook kun je de sha's van de
beide commits terugvinden. Als je een push hebt gedaan naar Github zie
je dat {\tt master} gelijk is aan {\tt remotes/origin/master}. Bekijk
wat er verder allemaal te zien is in {\tt gitk} en sluit de
applicatie. \\ 

\begin{figure} 
\begin{center}
  \caption{Gitk geeft een grafische weergave van de history}
  \includegraphics[width=\textwidth]{pictures/gitk-1.png}
  \label{fig:gitk-1}
\end{center}
\end{figure}

\step Ga nu verder vanaf de volgende webpagina. Het vormt een herhaling
van wat we hierboven al gedaan hebben, maar het kan geen kwaad om het
nog eens te doen en het nog eens op je in te laten werken.
\url{http://book.git-scm.com/3_normal_workflow.html} \\

\step Branching. Een van de bijzonderheden van {\tt git} is dat je
kosteloos lokaal een branch kan maken van je repository, waarin je
even kan experimenteren. Mocht het experiment op niks uitlopen, dan
verwijderen je gewoon de branch. Mocht er iets succesvols uitkomen,
dan kun je jouw experiment met de master-branch mergen. 
De volgende pagina gaat daar over. Lees de
pagina en volg de instructies:
\url{http://book.git-scm.com/3_basic_branching_and_merging.html}. Je
vraag je misschien af waar {\tt git reset --hard} voor staat. Lees dan
deze pagina nog door: \url{http://progit.org/2011/07/11/reset.html}. \\

\step We gaan nu met meerdere mensen aan hetzelfde project werken,
zoals je ook gewend bent in een themaopdracht. Je moet eerst iemand
toegang geven in Github als collaborator. Ga naar je repository,
selecteer ``Admin'' en vervolgens ``Collaborators''. Kies dan de naam
van je Github-partner en klik op Add. Jouw Github-partner krijgt nu
een bericht in zijn inbox dat hij push-rechten heeft tot jouw
repository.

\step We werken nu verder op het systeem van je Github-partner. Als je
alleen werkt kun je dit faken door het project zometeen in een andere directory
te clonen. Wel moet je inloggen met je tweede account bij Github om te
simuleren alsof je met zijn tweeën werkt. De voorkeur heeft het zeker om een ander systeem dan
waarmee we zojuist hebben gewerkt te gebruiken, om het samenwerken aan een
themaopdracht beter te kunnen simuleren. Als je nog geen {\tt git} hebt
geïnstalleerd op het andere systeem moet je dat eerst doen.

Log in met het account van je Github-partner bij Github en
bezoek dan de repository die zojuist is aangemaakt. Je bezoekt dus een
repository van een ander account dan waarmee je nu bent ingelogd, maar
die jou wel push-rechten heeft gegeven. Kopieer de git-link. Deze ziet
er ongeveer zo uit: {\tt git@github.com:<naam>/les3.git}. Open nu
bash en {\tt cd} naar een directory waar je de repo kan neerzetten. Vervolgens
gaan we in deze directory de repository clonen: {\tt git clone} gevolgd door
de gekopieerde git-link. \\

\step Nu gaan we wat wijzigen verrichten aan onze lokale repository.
Open file1, maak de file leeg en typ op de eerste regel: {\tt tekst
  systeem partner}. Sla de file op en typ vervolgens {\tt git add file1} gevolgd door
{\tt git commit -m 'een regel tekst
  toegevoegd'}. \\

\step Typ nu: {\tt git log}. Je ziet nu informatie over de drie
commits. Onze lokale repo loopt één commit voor op origin/master, de
repo op Github. \\

\step Haal de sha van de commit uit de logs van voor de wijziging aan
file1. En typ dan: {\tt git reset <sha>}. Je krijgt nu van {\tt git}
de melding dat er unstaged changes zijn na de reset. Dit betekent dat
file1 nog steeds de wijzigingen bevat van zojuist, maar dat deze
wijziging niet niet gestaged is, d.w.z. nog niet is toevoegd aan de
index. Je kunt dus een reset doen, zonder de files in je working
directory te wijzigen. \\

\step Om de wijziging écht terug te draaien zullen we de optie {\tt
  --hard} moeten meegeven aan {\tt git reset <sha>}. Probeer het dus
nogmaals, maar nu met {\tt git reset <sha> --hard}. \\

\step Bekijk de inhoud van file1. Je kunt nu zien dat de wijziging is
teruggedraaid. Wees dus voorzichtig wanneer je {\tt --hard} gebruikt,
je files in de working directory kunnen hierdoor wijzigen. Zorg er
wederom voor dat file1 leegt wordt en voeg daarna op de eerste regel
weer {\tt tekst systeem partner} toe. Doe vervolgens {\tt git commit -a -m
  'wijziging in file1'}. De optie {\tt -a} geeft aan dat je gewoon
alle wijzigingen in bestaande files (niet van nieuwe files) wil stagen
en committen. \\ 

\step Doe een {\tt git status}. Je krijgt nu de melding dat onze
lokale repository één commit voorloopt op origin/master. Nu gaan we
onze wijziging doorsluizen naar Github. Dit kan met {\tt git
  push origin master}. Kijk op Github of de wijziging is doorgekomen. \\

\step Stap nu weer over naar het andere systeem waarmee we begonnen.
Maak ook hier file1 leeg en voeg op de eerste regel de tekst {\tt
  tekst mijn eigen systeem} toe, en add+commit deze wijziging. Probeer dan
een {\tt git push}. Als het goed is weigert {\tt git}deze wijziging. We
moeten namelijk van {\tt git}eerst een pull doen (volgende stap). Dit is het omgekeerde van
push: we gaan wijzigingen van de Github-repo (origin/master) naar onze
eigen repo halen. \\

\step Doe {\tt git pull}. {\tt Git} haalt nu de wijzigingen vanaf Github
naar onze lokale repo. Omdat er nu een conflict is op de eerste regel
van file1, zullen we handmatig moeten bepalen wat we hiermee doen.
Toevoegingen of verwijderingen kan {\tt Git} automatisch oplossen, maar
wijzigingen op dezelfde regel niet. De inhoud
van file1 zal er nu ongeveer zo uit zien:

\begin{minted}{raw}
<<<<<<< HEAD
tekst mijn eigen systeem
=======
tekst systeem partner
>>>>>>> e70c12d8fc8e7f0eae9cf43405ea32741f7bbd15
\end{minted}

De tekst tussen {\tt <<<<<<< HEAD} en {\tt =======} is de tekst van
revisie HEAD, de huidige revisie dus. De tekst tussen {\tt =======} en
{\tt >>>>>>> e70c12d8fc8e7f0eae9cf43405ea32741f7bbd15} (waarbij de sha
heel waarschijnlijk zal verschillen van de jouwe!) is afkomstig van
jouw Github-partner. We moeten nu kiezen wat we gaan doen.
Als we besluiten om beide regels te houden, dan moeten we de rest van
de tekst die het conflict aangeeft verwijderen en opnieuw een
add+commit doen. De tekst in file1 wordt nu dus: 

\begin{minted}{raw}
tekst mijn eigen systeem
tekst systeem partner
\end{minted}

Doe vervolgens weer een add+commit. \\

\step Push de wijziging weer naar Github en kijk of het goed is
aangekomen. Bekijk ook de history van de file. Je kunt de wijzigingen
per commit terugkijken. \\

\step Ga vervolgens weer naar het systeem van de Github-partner. Doe
daar een {\tt git pull}. Vervolgens zijn beide collaborators up to
date. Doe een {\tt gitk} om een grafische representatie te bekijken
van de commits. Maak daarvan een screenshot en plak deze in je
portfolio met een toelichting erbij. \\

\step Op een bepaald moment gedurende de ontwikkeling van een project
wil je soms een (herkenbare) versie vastleggen. Niet alleen voor
ontwikkelaars zelf is het handig om over versie 1.2 of 0.5-beta te
praten, maar ook voor de buitenwereld is het handig om te weten welke
versie van een project bruikbaar is in een bepaalde context. Daarvoor
kun je in {\tt git} tags gebruiken. Tags corresponderen met een
bepaalde commit-sha van het project. Stel dat we nu bepaald hebben dat
ons project les3 klaar is om als versie 0.1 de wereld in te gaan, dan
kunnen we dat aan git vertellen als volgt: {\tt git tag -a v0.1 -m
  'version 0.1'}. Om de tag zichtbaar te krijgen in Github moet je bij
push nog een extra optie meegeven: {\tt git push --tags}. Probeer dit
uit. Kijk nu onder het tabje tags van het project op Github. Daar zie
je als het goed is de tag. Ook is er bij elke tag de optie om die
versie van het project te downloaden als zip: handig voor mensen die
wel jouw code willen hebben, maar even geen {\tt git} (willen)
gebruiken. Voor meer info over tagging zie
\url{http://learn.github.com/p/tagging.html}. 

We weten nu (net) genoeg om via {\tt git} en Github samen te kunnen werken
aan een project Nog wat laatste tips:

\begin{itemize}
\item Met {\tt git add .} voeg je alle wijzigingen, inclusief nieuwe
  files in één keer toe aan de index
\item Met {\tt git commit -a} voeg je alle wijzigingen toe aan de
  index én doe je een commit. Pas op: nieuwe files worden hierbij
  niet meegenomen. In dat geval moet je eerst {\tt git add .}
  doen.
\item Het is zeer aan te bevelen de documentatie over tags,
  .gitignore, rebasing, etc. een andere keer eens rustig door te nemen. Ga in dat
  geval verder vanaf de webpagina
  \url{http://book.git-scm.com/3_git_tag.html}. 
\item Als je geen contributor bent of mag worden van een bepaald
  project op Github, kun je altijd het project forken. Er ontstaat dan
  een kopie van het andere project onder jouw Github-account en daar
  kun je dan je eigen ding in doen. Mocht je iets heel cools hebben
  gemaakt, of bepaalde bugs hebben opgelosts, dan kan je een pull
  request indienen bij het oorspronkelijke project. Zij ontvangen dan
  een bericht waarin ze de keuze hebben om wel of niet jouw
  wijzigingen te accepteren.
\item Voor integratie met {\tt git} en Eclipse is er een
  Eclipse-plugin beschikbaar, genaamd {\tt EGit}. Zie \url{http://eclipse.org/egit/}.
\end{itemize}



\subsubsection{Links}
\url{http://www.github.com} \\
\url{http://book.git-scm.com/} \\
\url{http://progit.org/book/} \\
\url{http://gitref.org/} \\
\url{http://eclipse.org/egit/}

%%% Local Variables: 
%%% mode: latex
%%% TeX-master: "dictaat"
%%% End: 
